\documentclass[12pt, letterpaper, titlepage]{article}
\usepackage[left=3.5cm, right=2.5cm, top=2.5cm, bottom=2.5cm]{geometry}
\usepackage[MeX]{polski}
\usepackage[utf8]{inputenc}
\usepackage{graphicx}
\usepackage{enumerate}
\title{Coś o sobie}
\author{Piotr Zienowicz}
\date{Październik 2022}
\begin{document}
\maketitle
\textbf{Przepis na pierogi z jagodami}
\textbf{Czas przygotowania: 50min}
\section{Potrzebne składniki:}
\textit{Składniki}
\\ 1 szczypta cukru waniliowego\\
\\ 500g mąki pszennej\\
\\ 1 szklanka ciepłej wody\\
\\ 20g oleju roślinnego\\
\\ 1 jajko\\
\\ 2 szczypty soli\\
\\ 300g jagód\\
\\ 3 szczypty cukru\\
\\ 1 kubek kwaśnej śmietany 18\\
\\ 2 szczypty cukru\\
\section{Przygotowanie}
\begin{enumerate}[a)]
\item{Zagnieść ciasto pierogowe.}
\item{Podzielić na kilka części (przykryć ściereczką lub miską by nie wyschło), a jedną z nich cienko rozwałkować podsypując niewielką ilością mąki.}
\item{Wykroić szklaneczką krążki, lekko rozciągnąć palcami, nałożyć owoce posypane cukrem na wilgotniejszą stronę, złożyć na pół, starannie skleić brzegi.}
\item{Gotowe pierożki ułożyć na ściereczce obsypanej mąką.}
\item{Porcjami wrzucać pierogi do wrzącej, lekko osolonej wody i delikatnie zamieszać drewnianą łyżką (garnek przykryć pokrywką).}
\item{Kiedy po kilku minutach pierogi wypłyną na wierzch zdjąć pokrywkę i dogotować je jeszcze 2-3 minuty.}
\item{Wyjąć łyżką cedzakową, przełożyć do miski z zimną wodą i ponownie przy pomocy cedzaka wyjąć.}
\item{Gotowe pierożki przełożyć na talerze i posypać cukrem lub polać kwaśną śmietaną ubitą z cukrem i cukrem wanilinowym.}
\end{enumerate}
\end{document}
