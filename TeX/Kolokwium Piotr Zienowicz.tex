\documentclass[12pt, letterpaper, titlepage]{article}
\usepackage[left=2.5cm, right=2.5cm, top=2.5cm, bottom=2.5cm]{geometry}
\usepackage[MeX]{polski}
\usepackage[utf8]{inputenc}
\usepackage{graphicx}
\usepackage{enumerate}
\title{Kolokwium pierwsze}
\author{Piotr Zienowicz}
\date{6.12.2022}
\begin{document}
\maketitle
\section{Szyfr AES}
Standard AES (\textit{Advanced Encryption Standard}) opublikowany został przez NIST w 2001
roku. AES jest szyfrem blokowym, zaprojektowanym w celu zastąpienia szyfru DES jako
przyjęty standard w różnorodnych zastosowaniach. W porównaniu z szyframi z kluczami
publicznymi - jak RSA, AES (jak zresztą większość szyfrów symetrycznych) jest nieporównanie
bardziej skomplikowany i nie da się wyjaśnić tak prosto jak wiele innych algorytmów
kryptograficznych. Względna prostota tej wersji umożliwia ręczne wykonywanie szyfrowania
i deszyfracji, dzięki czemu łatwiej można zrozumieć detale pełnego algorytmu AES \cite1.
\subsection{Artytmetyka ciał skończonych}
W Algorytmie AES wszystkie obliczenia wykonywane są na 8-bitowych bajtach, dodawanie,
mnożenie i dzielenie wykonywane są w ciele skończonym GF($2^8$). Mówiąc skrótowo, ciało to
zbiór, w którym dodawanie, odejmowanie, mnożenie i dzielenie są działaniami wewnętrznymi:
wynik dowolnej z tych operacji wykonanej na elementach zbioru również należy do tego zbioru.
Dzielenie definiowane jest jako mnożenie przez element odwrotny: a/b oznacza to samo,co a($b^{-1}$).\\
\\
Wszystkie niemal algorytmy szyfrujące, zarówno te konwencjonalne, jak i te z kluczami publicznymi, obejmują obliczenia na liczbach całkowitych. Jeżeli jedną z operacji jest dzielenie, wymagana jest arytmetyka zdefiniowana nad ciałem: wykonalność dzielenia wiąże się bowiem z wymaganiem, by każdy nie zerowy element ciała posiadał odwrotność multiplikowaną. Ze względu na wygodę obliczeń oraz efektywności implementacji chcielibyśmy operować liczbami dającymi się zapisywać na ustalonej (n) liczbie bitów i jednocześnie w pełni wykorzystującymi zakres wartości (od 0 do $2^{n-1}$) oferowanych przez tę liczbę bitów. Niestety, zbiór $Z_{2n}$ ze zwykłą arytmetyką modulo $2^n$ nie jest ciałem, ponieważ żadna liczba parzysta nie posiada w tej arytmetyce odwrotności multiplikowanej nie dla każdego elementu b $\in$ $Z_{2n}$ istnieje takie x, że bx mod $2^n$ = 1.\\
\\
Możliwe jest jednak takie zdefiniowanie działań w zbiorze $Z_{2n}$, by stał się ciałem GF($2^n$). Rozpatrzymy zbiór S wielomianów stopnia nie wyższego niż n - 1 ze współczynnikami binarnymi (czyli pochodzącymi ze zbioru {0,1}). Każdy z tych wielomianów ma postać\\
\\
$$f(x) = a_{n-1}x^{n-1} + a_{n-2}x^{n-2} + ... + a_{1}x + a_{0} = \sum^{n-1}_{i=0}a_{i}x^{i}$$\\
\\
gdzie każde $a^i$ ma wartość 0 albo 1. W zbiorze S istnieje wówczas dokładnie $2^n$ różnych wielomianów; dla n = 3 jest to 8 następujących wielomianów:
$$
\begin{array}{llllll}
0 & x & x^2 & x^2 + x & \\
1 & x + 1 & x^2 + 1 & x^2 + x & 1\\
\end{array}$$\\
\\
Można sprawić, że zbiór S będzie ciałem skończonym, definiując odpowiednio operacje na jego elementach:
\begin{enumerate}
\item Operacje te opierać się muszą generalnie na tradycyjnych regułach arytmetyki wielomianowej, jednak z zastrzeżeniami wymienionymi w punktach 2. i 3.
\item Operacje na współczynnikach wielomianów wykonywane są modulo 2; dodawanie jest wówczas równoważne operacji XOR. 
\item Gdy wynikiem mnożenia wielomianów jest wielomian f(x) stopnia wyższego niż n -1, należy wielomian ten zredukować modulo pewien nieredukowalny wielomian m(x) stopnia n - czyli zamiast oryginalnego wielomianu f(x) wziąć resztę r(x) z jego dzielenia przez m(x), oznaczoną f(x) mod m(x). Wielomian nazywamy \textbf{nieredukowalnym}, jeżeli nie można go przedstawić w postaci iloczynu dwóch wielomianów niższych stopni.
\end{enumerate}
\begin{table}[h]
\centering\caption{Parametry algorytmu AES \cite2}
\begin{tabular}{|l|c|c|c|}
\hline
Długość klucza(w słowach/bajtach/bitach) &4/16/128 & 6/24/192 & 8/32/256  \\
\hline
Rozmiar bloku wejściowego (w słowach/bajtach/bitach) & 4/16/128 & 4/16/128& 4/16/128\\
\hline
Liczba rund & 10 &12& 14\\
\hline
Długość podklucza rundy (w słowach/bajtach/bitach)& 4/16/128 & 4/16/128 &4/16/128 \\
\hline
Rozmiar rozwiniętego podklucza (w słowach/bajtach) &44/176& 52/208& 60/240\\
\hline
\end{tabular}
\end{table}
\newpage
\section{Ciasto z jabłkami i żurawiną}
Ciasto z jabłkami i żurawiną \cite3 to fantastyczna szarlotka dla fanów żurawiny. Połączenie jabłek i świeżej żurawiny jest bezbłędne. Ciasta nie trzeba chłodzić i wstępnie podpiekać, dlatego przygotowanie nie powinno nam zająć dużo czasu. Świeża żurawina jest obecnie w sezonie więc korzystajmy z tego dobrodziejstwa, choć w przepisie bez problemu i żadnych dodatkowych zmian można użyć żurawiny mrożonej. Polecam!
\section*{Składniki na ciasto}
\begin{enumerate}[•]
\item 300 g mąki pszennej
\item 200g masła
\item 1 łyżeczka proszku do pieczenia
\item 1 łyżeczka zmielonego cynamonu
\item 1 duże jajko
\item 1 żółtko, z dużego jajka
\item 110 g jasnego miałkiego brązowego cukru
\end{enumerate}
Wszystkie składniki na ciasto umieścić w misie malaksera i zmiksować do połączenia. Otrzymane ciasto będzie bardzo gęste, wręcz „ciasteczkowo” gęste. Jeśli ciasto będzie mocno klejące można je schłodzić w lodówce przez około 60 minut.\\
\\
Kwadratową formę o boku 24 cm wyłożyć papierem do pieczenia, sam spód. Ciasto podzielić na 2 równe części. Pierwszą częścią ciasta wylepić spód formy, na nie równomiernie wyłożyć nadzienie jabłkowo – żurawinowe, a na górę wyłożyć drugą część ciasta – w kawałeczkach.\\
\\
Ciasto z jabłkami i żurawiną piec w temperaturze 180ºC, najlepiej bez termoobiegu (czyli z grzaniem góra-dół), przez około 50 minut. Po wystudzeniu można oprószyć dodatkowym cukrem pudrem.
\\
\section*{Składniki na nadzienie jabłkowo - żurawinowe}
\begin{enumerate}[•]
\item 600 g jabłek odmiany szarlotkowej
\item 300 g żurawiny
\item 1/4 szklanki cukru (lub mniej, do smaku)
\item 1 łyżeczka zmielonego cynamonu
\item 1 łyżka soku z cytryny
\item 1 łyżka skrobi ziemniaczanej
\end{enumerate}
Jabłka umyć, obrać, usunąć gniazda nasienne a następnie pokroić w półplasterki.\\
\\
Pokrojone jabłka umieścić w garnku, dodać żurawinę, cukier, cynamon i sok z cytryny, następnie
wymieszać. Pogotować przez kilka minut do wstępnego odparowania wody i popękana
żurawiny. Następnie oprószyć przez sitko skrobią ziemniaczaną i wymieszać. Zdjąć z palnika, nie trzeba studzić (można gorące wyłożyć na przygotowane ciasto w formie).
\newpage
\begin{thebibliography}{1}
\bibitem{1}
William Stallings.
\textit{Kryptografia i bezpieczeństwo sieci komputerowych Helion}, 2012, str 198-200.
\bibitem{2}
William Stallings.
\textit{Kryptografia i bezpieczeństwo sieci komputerowych Helion}, 2012, str 204.
\bibitem{3}
https://mojewypieki.com/przepis/ciasto-z-jablkami-i-zurawina
\end{thebibliography}
\end{document}